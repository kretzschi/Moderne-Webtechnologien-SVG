\section{Aufbau und Syntax}
\subsection{Einleitung}
In diesem Abschnitt werden die wichtigsten Strukturen zum Aufbau eines SVG Dokumentes vorgestellt.
 \subsection{<svg>}
 Das \texttt{<svg>}-Tag ist das Grundlegenste. Innerhalb dessen werden alle (grafischen) Eigenschaften eines Dokuments definiert und das beliebig komplex, wobei \texttt{<svg> </svg>} bereits gültig ist. Zu bemerken ist noch, dass durch Attribute einige Eigenschaften der Ergebnisdatei bestimmt werden könne. Zum Beispiel ist es möglich über die Attribute \texttt{width} und \texttt{height} die Breite respektive Höhe des Zeichenbereichs festzulegen.
 
 \subsection{<g>}
 Das Gruppenelement \texttt{<g>} kann zum Zusammenfassen grafischer Objekte verwendet werden. Das hat einmal den Vorteil, die Lesbarkeit von SVG Code zu verbessern und zusammenhängende Strukturen zu verdeutlichen.