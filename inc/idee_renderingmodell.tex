\section{Idee und Renderingmodell}
\subsection{Über SVG}
SVG wurde im September 2001 vom W3C veröffentlich und stellt eine empfohlene Spezifikation für zweidimensionale Vekorgrafiken bereit. Die aktuelle Version 1.1 wurde als W3C Recommendation am 14.Januar 2003 veröffentlicht und dient als Grundlage dieser Ausarbeitung \cite{svg:2003}.\\
Sie basiert auf der Auszeichnungssprache XML \cite{xml:2008}.\\
In diesem Kapitel soll kurz die Idee hinter SVG, sowie dessen Renderingmodell skizziert werden.

\subsection{Grundlegende Konzepte}
Hinter der Abkürzung SVG, also Scalable Vector Graphics (dt. skalierbare Vektorgrafiken), stehen mehrere Ideen und grundlegende Konzepte, die mit dem Standard umgesetzt werden sollen.\\

\textit{Skalierbar}\\
Der Begriff skalierbar bezieht sich auf zwei Dimensionen. Zum einen ist es eine Eigenschaft der Grafiken. Unabhänging der möglichen Auflösung eines Endgerätes, soll diese Art der Grafiken immer gleich scharf abgebildet werden. Auch bei Vergrößerungs- und Verkleinerungseffekten gibt es bei Vektorgrafiken keinen Qualitätsverlust.
Der zweite Aspekt betrifft die Skalierbarkeit im Web allgemein, da es mit anderen Standards problemlos integriert werden kann.\\

\emph{Vector}\\

 
\subsection{Renderingmodell}
Das Renderingmodell von SVG kann grundlegend mit drei Begriffen zusammengefasst werden. Natürlich könnte man dieses Thema umfangreicher ausführen, jedoch sollen an dieser Stelle die Grundideen ausreichen.

\emph{Paint}\\
Paint kann hier mit "Anstrich" übersetzt werden.
\emph{Reihenfolge}\\
\emph{Effekt}\\