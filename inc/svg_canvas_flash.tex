\section{SVG vs. Canvas vs. Flash}
Die Recommendation zum SVG ist natürlich nicht die einzige Technologie, die im Browser Standbilder oder bewegte Bilder bis hin zu Videos oder Spielen darstellen kann. Jedoch ist offensichtlich, dass SVG Unterschiede zu den herkömmlich genutzten Technologien, wie zum Beispiel Canvas oder Flash ausweisen muss.
\subsection{SVG vs Canvas}
Zwischen SVG und Canvas ist der größte Unterschied in der Zeichnung des Bildes wieder zu finden. SVG sind, wie oben schon erwähnt, Vektorgrafiken, die als Objekte im XML Format abgespeichert werden. Somit kann unabhängig vom Zoom das Bild immer glatt gerendert und unverpixelt dargestellt werden.\\
Falls es dazu einen Gegensatz gibt, dann wäre es die Technik die auf dem Canvas verwendet wird. Hierbei ist die Darstellung vollkommen objektunabhängig. Die einzelnen Pixel werden je nach Anweisung mit ihrer bestimmten Farbe an einen bestimmten Ort gesetzt und danach verliert das Canvas jegliche Information über dessen Herkunft. Natürlich kann ein Framework sämtliche Daten, die zu den Pixeln gehören, in einer eigenen Datenstruktur speichern und weiter verarbeiten, jedoch gibt es dafür keine Möglichkeiten im Canvas.\\
Das wird auch deutlich, wenn die Techniken in ihrer Performanz gegenübergestellt werden. Das SVG ist wesentlich performanter, wenn es mit einem oder wenigen Objekten arbeiten muss, welche lediglich durch Bildmanipulationen oder Transformationen bearbeitet wird. Eine Karte ist dafür ein gutes Beispiel. Diese kann einmal gerendert werden und danach reicht es beim SVG aus, weitere Darstellungen durch Translation und Transformation zu realisieren. Dahin gehend müsste ein Canvas bei jedem Zoomeffekt wieder neu gemalt werden.\\
Ein anderes Szenario ist, wenn viele Objekte bearbeitet und dargestellt werden müssen. Hierbei wird das SVG sehr langsam, was auf mehrere Ursachen zurück geführt werden kann. Eine mögliche Ursache ist, dass die XML Struktur des SVGs viel zu sperrig wird, da jedes Objekt in Struktur eingefügt werden muss und dann das ganze Bild neu gerendert wird. Auf der anderen Seite, also beim Canvas, wird dieses Szenario schneller vollzogen. Da die Technologie nicht die einzelnen Objekte speichert, kann diese in gleichbleibender Geschwindigkeit das Bild darstellen, ohne von der Anzahl der Objekte beeinflusst zu werden.
\subsection{SVG vs Flash}
SVGs und Flash ist in vielen Punkt recht ähnlich aufgebaut. Beide Technologien basieren auf Vektorgrafiken, es kann gescriptet werden, durch Javascript oder Actionscript und es gibt verschiedene Möglichkeiten Animationen einzufügen.\\
Trotzdem es auf dem ersten Blick recht ähnlich aussieht, gibt es doch gravierende Unterschiede. Der erste Unterschied bezieht sich auf das Mitwirkungsrecht der Öffentlichkeit: SVG ist ein offener Standard, während Flash nur eine Technologie von Adobe ist. Dieser Fakt wirkt sich auf verschiedene Ebenen aus:
\begin{enumerate}
  \item An dem SVG dürfen alle Mitglieder der W3C mitbestimmen, wogegen beim Flash lediglich Adobe auf ihre Mitarbeiter und das Feedback einer Kernnutzerschaft zurück greift
  \item SVGs können mit einem freien Editoren bearbeitet werden. Man muss dafür nicht die gewohnte Editorumgebung ändern. Dem gegenüber steht Flash, dass zwar auch durch freie Editoren bearbeitet werden kann, aber diese müssen erst heruntergeladen und eingerichtet werden und stehen den kommerziellen Editoren von Adobe in ihrer Funktionalität weit hinterher.
  \item Die Kompatibilität zu anderen Standards spielt dabei auch eine Rolle. SVGs wurden dafür konzipiert, kompatibel mit anderen Technologien, wie XML, Javascript oder HTML zu sein. Flash hingegen ist eine abgeschlossene Sprache, wofür erst andere Standards angepasst werden müssen, damit diese im Flash benutzt werden können, wie zum Beispiel Actionscript.
\end{enumerate}
Ein weiterer Vorteil für das SVG ist in dem Funktionalitätsumfang des Standards zu finden. Denn wie oben erwähnt, kann das SVG Filter- und Maskingeffekte darstellen. Dem gegenüber steht Flash, das dafür keine Funktionalität zur Verfügung stellt
