\section{Vergleich zwischen SVG Tiny und SVG Mobile}
SVG Tiny und SVG Mobile sind zwei Profile der SVG Full Recommendation. Das heißt, diese beiden Bibliotheken haben Schnittmengen vom SVG, sind jedoch deutlich kleiner als die vollständige Bibliothek. Der Bedarf dazu kam von der Industrie. Diese nutzte das SVG schon innerhalb ihrer Software und brauchten noch eine Lösung für mobile Endgeräte. Anfang der 200er waren die mobilen Endgeräte nicht leistungsfähig genug, um den vollen Umfang des SVG's darstellen zu können. Hinzu kommt, dass die Bibliothek für damalige Verhältnisse einen großen Speicherplatz bedarf. Um diesen Bedarf gerecht zu werden, wurden diese zwei Profile entwickelt.\\

Interessanterweise wird das SVG momentan in der Version 1.1 empfohlen, während das SVG Tiny in der Version 1.2 vorhanden ist. Der Grund dafür liegt im Entwicklungsprozess der SVG Recommendations. Diese wurde weiter entwickelt, sodass einige wenige Fehler verbessert wurden. Aber der neue Funktionsumfang war nicht groß genug, dass sich dafür eine neue Iteration gelohnt hätte. Als Folge entstand lediglich eine neue Version für das SVG Tiny, während die minimalen Verbesserungen in die SVG 1.1 Version einflossen.\\

Die Abgrenzung von SVG Tiny und SVG Mobile ist nicht so gravierend. Das SVG Tiny und das SVG Base ergeben zusammen das SVG Mobile. Somit ist das Mobile eine Übermenge und das SVG Full ist eine Übermenge über alle Profile plus mehr Fähigkeiten. Warum dabei eine strikte Trennung zwischen Tiny und Mobile vorgenommen wurde, ist nicht mehr ersichtlich. Jedoch können Anwendungen mit dem Tiny programmiert werden und weiterhin im Mobile oder SVG Full benutzt werden, da alle Funktionen aufwärtskompatibel sind.\\

Es ist in der Recommendation nicht deutlich heraus getragen worden, welche Teile der Bibliothek exakt ausgelassen wurden, jedoch beim Vergleich beider Bibliotheken stellt sich heraus, dass sogar wenige Merkmale zusätzlich im Tiny vorhanden sind. Zum Beispiel kann man mit dem Tiny multimediale Elemente einfügen. Dieser Unterschied ist darauf zurückzuführen, dass die Tiny in der 1.2 Version ist und die SVG Full in der 1.1 Version ist. Dieses Merkmal ist dann für die SVG 2.0 vorgesehen.\\
Weiterhin kann Geo Lokalisation verwendet werden. Das ist gerade bei mobilen Anwendungen ein sehr elementares Merkmal, dass die Anwendungen auf die momentane Umgebung eingehen kann, in dem über das Internet oder das Mobilfunknetz der momentane Standort erfahren wird. Somit werden Anwendungen zur Navigation durch Landkarten erst möglich.\\
Das waren die Elemente die hinzu gekommen sind. Jedoch wie schon anfänglich erwähnt, wurde die Bibliothek eher verkleinert. Dazu wurden größtenteils die sehr rechenintensiven Elemente heraus genommen, zum Beispiel Clipping, Masking und Filtereffekte. Dass diese Elemente sehr aufwendig in Bezug auf Leistung und Speicherkapazität sind, ist offensichtlich. Jedoch ist einsehbar, dass die grundlegenden Merkmale ausgereicht haben, denn dieser Standard wurde viele Jahre beibehalten.
